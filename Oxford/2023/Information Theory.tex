\documentclass{styles/tufte}
\usepackage{styles/graph}

\course{B8.5 Graph Theory}
\courseterm{MT 2023}
\author{Jiaming (George) Yu}
\email{jiaming.yu@jesus.ox.ac.uk}
\date{\today}

\begin{document}

\maketitle
\tableofcontents
\newpage



\section{Group Theory Basics}


\subsection{Basic Groups}

  \subsubsection{Cyclic Groups}

    \begin{definition}{Cyclic group}{}
      The \term{cyclic group} $C_n$ is the group generated by a non-trivial generator $g$ of order $n$.
    \end{definition}
    
    There are two common realizations of a cyclic group. The first is the additive group of integers modulo $n$ (i.e. $C_n \cong \ZZ_n$). The second is the multiplicative group with powers of a primitive $n$\tsp{th} root of unity $\xi \in \CC$.
  
  
  \subsubsection{Symmetric Groups}
    
    \begin{definition}{Symmetric group}{}
      The \term{symmetric group} $S_n$ is the group of all permutations (bijections) of the set $\set{1, \dots, n}$ with composition as the group operation.
    \end{definition}


\end{document}
