\documentclass{styles/tufte}
\usepackage{styles/analysis}

\course{A2: Metric Spaces and Complex Analysis}
\courseterm{MT 2022}
\author{Jiaming (George) Yu}
\email{jiaming.yu@jesus.ox.ac.uk}
\date{\today}

\begin{document}

\maketitle
\tableofcontents
\newpage



\part{Metric Spaces}

\section{Metric Spaces}

Metrics provide a generalization to the notion of distance.

\begin{definition}{}{}
  A \term{metric space} is an ordered pair $(X, d)$, where $X$ is a set and $d: X \times X \to \RR$ a function called the \term{metric} with the following properties:
  \begin{romanenum}
    \item Positivity: $\forall x, y \in X: d(x, y) \geqslant 0$ with $d(x, y) = 0 \iff x = y$
    \item Symmetry: $\forall x, y \in X: d(x, y) = d(y, x)$
    \item Triangle ineqality: $\forall x, y, z \in X: d(x, y) + d(y, z) \geqslant d(x, z)$
  \end{romanenum}
\end{definition}

Note that a set can be paired with different metrics to form different metric spaces. For example, $\RR^n$ can be paired with any $p$-norm-induced metric. Other examples of metric include the discrete metric, where the metric is $1$ if $x = y$ and $0$ otherwise, and the hamming distance. It is often the case that the metric won't be specified when talking about a metric space. In such cases, it is assumed to use the canonical metric for the given set.

\begin{definition}{Product space}{}
  Let $(X, d_X), (Y, d_Y)$ be metric spaces. Their product space $X \times Y$ consists of ordered pairs $\set{(x, y): x \in X, y \in Y}$ with metric $d_{X \times Y}$ defined as
  \[ d_{X \times Y}((x_1, y_1), (x_2, y_2)) = \sqrt{d_X(x_1, x_2)^2 + d_Y(y_1, y_2)^2} \]
\end{definition}

%For example, consider \term{$\ell^p$ spaces}, which are defined as
%
%the canonical norm defined
%\[ \norm{x}_p = \Big(\sum_i \abs{x_i}^p\Big)^\frac{1}{p} \]
%The $1$-norm and $2$-norm are the taxicab distance (a.k.a.~Manhattan distance) and Euclidean distance respectively. We can pair any $\RR^n$ with any metric given by $p$-norm\marginnote{Note that norms and metrics are different concepts, but they share much in parallel} to obtain a metric space.


\subsection{Norms}
  
  Norms are a more general concept than metrics, and all metric can be induced from a norm. It could be said that norm generalizes metric in the same way that length generalizes distance.
  
  \begin{definition}{}{}
    Let $V$ be a vector space over $\RR$. A function $\norm{\funcarg}: V \to [0, \infty)$ is called a \term{norm} if it satisfies the following:
    \begin{romanenum}
      \item $\norm{x} = 0 \iff x = 0$
      \item $\forall \lambda \in \RR, x \in V: \norm{\lambda x} = \abs{\lambda} \cdot \norm{x}$
      \item $\forall x, y \in V: \norm{x + y} \leqslant \norm{x} + \norm{y}$
    \end{romanenum}
  \end{definition}
  
  Given any norm, it is indeed the case that $d(x, y) \defeq \norm{x - y}$ defines a metric on $V$. Importantly, the converse is not true. For example, the discrete metric does not arise from any norm. In fact, metrics which arise from a norm are \em{translation- and scalar-invariant}\marginnote{That is, $d(x + z, y + z) = d(x, y)$ and $d(\lambda x, \lambda y) = \abs{\lambda} d(x, y)$ respectively.}, which are properties arbitrary metrics do not have.
  
  \term{Normed spaces} are vector spaces endowed with a norm. We can also think them as metric spaces whose metric arises from the norm.


\subsection{Balls and Boundedness}
  
  \begin{definition}{Balls}{}
    Let $X$ be a metric space. For $a \in X$ and $\epsilon > 0$, we define an \term{open ball} as
    \[ B(a, \epsilon) = \set{x \in X: d(x, a) < \epsilon} \]
    and a \term{closed ball} as
    \[ \overline{B}(a, \epsilon) = \set{x \in X: d(x, a) \leqslant \epsilon} \]
  \end{definition}



\section{Limits and Continuity}

\begin{definition}{Limit}{}
  Let $(x_n)_{n=1}^\infty$ be a sequence of elements in a metric space $(X, d)$. If for some $x \in X$ we have $\forall \epsilon > 0: \exists N \in \NN: \forall n \geqslant N: d(x_n, x) < \epsilon$, then $x$ is the \term{limit} of $(x_n)_{n=1}^\infty$, and we write $x_n \to x$ or $\lim_{n \to \infty} x_n = x$.
\end{definition}

By simple contradiction with triangle inequality, we can see that limits are unique.

\begin{definition}{Continuity}{}
  Let $(X, d_X), (Y, d_Y)$ be metric spaces. We say that a function $f: X \to Y$ is \term{continuous at $a$} for some $a \in X$ if the following is true
  \[ \forall \epsilon > 0: \exists \delta > 0: \forall x \in X: \left( d_X(a, x) < \delta \implies d_Y(f(a), f(x)) < \epsilon \right) \]
  We say $f$ is \term{continuous} if $f$ is continuous at every $a \in X$.
\end{definition}

\begin{lemma}{}{}
  Let $X, Y$ be metric spaces and $f: X \to Y$. Then $f$ is continuous at $a \in X$ if and only if $\lim_{n \to \infty} f(x_n) = f(a)$ for some sequence $(x_n)_{n=1}^\infty$ with $x_n \to a$. In short, $\lim_{x \to a} f(x) = f(a)$.
\end{lemma}

As with real analysis, we also have the notion of uniform continuity, where the choice of $\delta$ only depends on $\epsilon$ and not on $a$.

\begin{definition}{Uniform continuity}{}
  Let $(X, d_X), (Y, d_Y)$ be metric spaces. We say that a function $f: X \to Y$ is \term{uniformly continuous} if
  \[ \forall \epsilon > 0: \exists \delta > 0: \forall x, y \in X: \left( d_X(x, y) < \delta \implies d_Y(f(x), f(y)) < \epsilon \right) \]
\end{definition}

Regarding normed spaces, we have the following result:

% Aren't all linear functions continuous?
\begin{lemma}{}{}
  Let $V, W$ be normed spaces with norms $\norm{\funcarg}_V, \norm{\funcarg}_W$, and $f: V \to W$ be a linear map. Then $f$ is continuous if and only if the set $\set{\norm{f(x)}: \norm{x} \leqslant 1}$ is bounded.
\end{lemma}


\subsection{Function Spaces}
  
  \begin{definition}{Bounded function space}{}
    For some set $X$, define $B(X)$ to be the space of all functions $f: X \to \RR$ such that $f(X)$ is bounded.
  \end{definition}
  
  \begin{definition}{Continuous function space}{}
    For some metric space $X$, define $C(X)$ to be the space of all continuous functions $f: X \to \RR$.
  \end{definition}
  
  \begin{definition}{Continuous bounded function space}{}
    Let $X$ be a metric space. Define $C_b(X) \defeq C(X) \cap B(X)$ to be the space of continuous, bounded functions on $X$. Note $C_b(X)$ inherits the norm from $B(X)$ as $C_b(X) \leqslant B(X)$.
  \end{definition}
  
  Note that for any $X$, $B(X)$ and $C(X)$ are both\marginnote{Note that we are slightly abusing notation here} vector spaces over $\RR$. For $B(X)$, $\norm{\funcarg}_\infty$ is a norm due to boundedness. For $C(X)$, the operations are pointwise addition and multiplication. In general, neither one includes the other. There is one\marginnote{We say `one', but essentially all closed interval $[a, b]$ works} notable exception where $C([0, 1]) \subseteq B([0, 1])$. Therefore, $C([0, 1])$ can also inherit the norm $\norm{\funcarg}_\infty$, but it can also be equipped with different norms, such as the $L^1$ norm:\marginnote{Note the absolute value!! This ensures $\norm{f}_1 = 0 \implies f = 0$}
  \[ \norm{f}_1 = \int_0^1 \abs{f(t)} \dd{t} \]
  In parallel to the $\ell^p$-norms on $\RR^n$, we can also define $L^p$-norms ($p \geqslant 1$) on $C[0,1]$ by
  \[ \norm{f}_p = \left( \int_0^1 \abs{f(t)} \dd{t} \right)^{\hspace{-0.25em}\frac{1}{p}} \]



\section{Isometries, Homeomorphisms, and Equivalence}

When studying a structure, we are interested in the maps which preserve that structure.

\begin{definition}{Isometry}{}
  Let $(X, d_X), (Y, d_Y)$ be metric spaces. A function $f: X \to Y$ is an \term{isometry} if for any $x, y \in X$ we have
  \[ d_X(x, y) = d_Y(f(x), f(y)) \]
\end{definition}

\begin{definition}{Homeomorphism}{}
  Let $X, Y$ be metric spaces. A function $f: X \to Y$ is a \term{homeomorphism} if it is continuous, a bijection, and it has a continuous inverse.
\end{definition}

Isometries are maps which preserve distances. Homeomorphisms are a less strict subset of maps than isometries.



\section{Open and Closed Sets}

\begin{definition}{Open set}{}
  Let $X$ be a metric space. We say a subset $U \subseteq X$ is \term{open} in $X$ if
  \[ \forall y \in U: \exists \delta > 0: B(y, \delta) \subseteq X \]
\end{definition}

\begin{definition}{Closed set}{}
  Let $X$ be a metric space. We say a subset $F \subseteq X$ is \term{closed} in $X$ if its complement $X \setminus F$ is open in $X$.
\end{definition}

Naturally, every open ball in a metric space is an open set, and every closed ball in a metric space is a closed set.

Importantly, being closed is not equivalent to not being open --- in fact, there can be sets which are both open and closed, and those which are neither open nor closed. We have the following remarks:
\begin{itemize}
  \item In $\RR$, we have that $\RR$ is both open and closed, $(0, 1)$ is open but not closed, $[0, 1]$ is closed but not open, $(0, 1]$ is neither, and $\QQ$ is also neither.
  \item If $X$ is equipped with the discrete metric, then every subset is open
  \item All singleton sets are closed (they are just closed balls with radius $0$)
\end{itemize}

\begin{proposition}{}{}
  Let $X$ be a metric space. Then,
  \begin{romanenum}
    \item $X$ and $\emptyset$ are both open
    \item Any countable union of open sets are open
    \item Any finite intersection of open sets are open
  \end{romanenum}
\end{proposition}
\begin{proof}
  (ii): Let $U = \bigcup_{i \in I} U_i$ be a countable union of open sets $U_i$, and let $x \in U$. Then $x$ is in some $U_i$, which is open, so some $B(x, \delta)$ is contained in $U_i$ and thus $U$.
  
  (iii): Let $U = \bigcap_{i=1}^n U_i$ be a finite intersection of open sets $U_i$, and let $x \in U$. For each $U_i$ we have some $B(x, \delta_i) \subseteq U_i$. Let $\delta = \min_i \delta_i$, since the collection is finite we have $\delta > 0$ and $B(x, \delta) \subseteq B(x, \delta_i) \subseteq U_i$ for all $i$. Therefore $B(x, \delta) \subseteq U$.
  
  (i) is then just a special case of (ii) and (iii).
\end{proof}

We note that, in general, an arbitrary intersection of open sets need not be open. For example, $U_i = (-1/i, 1/i)$ are all open in $\RR$, but their intersection is $\set{0}$ which is not open.

For closed sets, we have the following by de Morgan's laws.

\begin{proposition}{}{}
  Let $X$ be a metric space. Then,
  \begin{romanenum}
    \item $X$ and $\emptyset$ are both closed
    \item Any countable intersection of closed sets are closed
    \item Any finite union of closed sets are closed
  \end{romanenum}
\end{proposition}


\subsection{Continuity and Open Sets}
  
  \begin{definition}{Neighborhood}{}
    Let $X$ be a metric space and $Z \subseteq X$. For some $z \in Z$, we say $Z$ is a \term{neighborhood} of $z$ if there is some $\delta > 0$ such that $B(z, \delta) \subseteq Z$.\marginnote{We see that if $Z$ is open then it is a neighborhood of any $z \in Z$, but we don't require a neighborhood to be open}
  \end{definition}
  
  \begin{proposition}{}{}
    Let $X, Y$ be metric spaces and $f: X \to Y$ be a map. Let $a \in X$, then $f$ is continuous at $a$ if and only if for every neighborhood $N$ of $f(a)$, the preimage\marginnote{Note here $f^{-1}$ is not a function! It is simply a shorthand and we are making no assumption of $f$ being invertible} $f^{-1}(N)$ is a neighborhood of $a$.
  \end{proposition}
  
  \begin{proposition}{}{continuity-open-sets}
    Let $X, Y$ be metric spaces and $f: X \to Y$ be a map. Then $f$ is continuous (on $X$) if and only if for each open subset $U \subseteq Y$, its preimage $f^{-1}(U)$ is open in $X$.
  \end{proposition}
  
  We can also get the same result for closed subsets (simply replacing every `open' in \cref{prop:continuity-open-sets} with `closed').
  
  Importantly, the above proposition does not say that continuous functions map open sets to open sets. An easy example is a constant function which is obviously continuous, but can map an open set to a closed set (singleton).
  
  
\subsection{Subspaces}
  
  For a metric space $X$, any $Y \subseteq X$ is a subspace with a restricted distance function. However, the notion of openness is not inherited.
  
  \begin{lemma}{}{}
    Let $X$ be a metric space and let $Y \subseteq X$. Then a subset $U_1 \subseteq Y$ is open in $Y$ if and only if there is an open subset $U_2 \subseteq X$ such that $U_1 = Y \cap U_2$.
    
    Similarly, $F_1 \subseteq Y$ is closed in $Y$ if and only if there is a closed subset $F_2 \subseteq X$ such that $F_1 = Y \cap F_2$.
  \end{lemma}
  
  As an example, ...





\section{Interiors, Closures, and Limit Points}

\begin{definition}{}{}
  Let $X$ be a metric space and let $S \subseteq X$.
  
  Define the \term{interior} $\interior(S)$ of $S$ to be the union of all open subsets of $X$ which are contained in $S$. The interior is always open.
  
  Define the \term{closure} $\closure(S)$ (sometimes also $\overline{S}$) to be the intersection of all closed subsets of $S$ containing $S$. The closure is always closed.\marginnote{In some sense, interior is the infimum and closure is the supremum.}
  
  Define the \term{boundary} $\partial S \defeq \closure(S) \setminus \interior(S)$.
  
  We say $S$ is \term{dense} if $\closure(S) = X$.
\end{definition}

As an example, $\QQ$ is dense in $\RR$, the set $\set{\frac{a}{2^n}: a \in \ZZ, n \in \NN}$ is also dense in $\RR$.


\subsection{Limit Points}
  
  \begin{definition}{Limit point}{}
    Let $X$ be a metric space and $S \subseteq X$. We say a point $a \in X$ is a limit point of $S$ if any open ball about $a$ contains a point of $S$ other than it self. Symbolically,
    \[ \forall r > 0: \exists x \in B(a, x): x \in S \land x \neq a \]
    We denote $L(S)$ for the set of all limit points of $S$.
  \end{definition}
  
  $L(S)$ is always closed.
  
  \begin{proposition}{}{}
    $\closure(S) = S \cup L(S)$
    
    Hence $S$ is closed if and only if $L(S) \subseteq S$.
  \end{proposition}



\section{Completeness}

\begin{definition}{}{}
  Let $X$ be a metric space. We say a sequence $(x_n)_{n=1}^\infty$ in $X$ is
  \begin{itemize}
    \item \term{bounded} if the set $\set{x_n: n \geqslant 1}$ is bounded
    \item \term{Cauchy} if $\forall \epsilon > 0: \exists N \in \NN: \forall n, m \geqslant N: d(x_n, x_m) < \epsilon$
    \item \term{convergent} if $\lim_{n \to \infty} x_n = a$ for some $a \in X$
  \end{itemize}
\end{definition}

\begin{definition}{Completeness}{}
  A metric space is \term{complete} if every Cauchy sequence converges.
\end{definition}

\begin{proposition}{}{}
  A subspace of a complete metric space is complete if and only if it is closed.
\end{proposition}

\begin{lemma}{Cauchy's intersection theorem}{}
  Let $X$ be a complete metric space and suppose $S_1 \supseteq S_2 \supseteq \cdots$ is a nested sequence of non-empty closed sets in $X$ with $\diameter S_n \to 0$ as $n \to \infty$. Then $\bigcap_{n=1}^\infty S_n$ contains a unique point $a$.\marginnote{Here, $\diameter S = \sup_{x, y \in S} d(x, y)$}
\end{lemma}


\subsection{The Contraction Mapping Theorem}
  
  \begin{definition}{Lipschitz map}{}
    Let $(X, d_X), (Y, d_Y)$ be metric spaces. We say a map $f: X \to Y$ is \term{Lipschitz} if there exists a constant $K \geqslant 0$ such that
    \[ \forall x, y \in X: d_Y(f(x), f(y)) \leqslant K d_X(x, y) \]
    
    When $X = Y$ and $K < 1$, we call $f$ a \term{contraction}.\marginnote{Note it is not sufficient that $d_Y(f(x), f(y)) < d(x, y)$}
  \end{definition}
  
  \begin{theorem}{Contraction mapping theorem}{}
    Let $X$ be a \em{non-empty}, \em{complete} metric space and suppose $f: X \to X$ is a contraction. Then $f$ has a unique fixed point $x \in X$ (i.e. where $f(x) = x$).
  \end{theorem}



\section{Connectedness}

\begin{definition}{Connectedness}{}
  A metric space is \term{disconnected} if it is the disjoint union of two non-empty open\marginnote{Note that openness here is w.r.t.~$X$, e.g. $[0, 1]$ is open in $[0, 1] \cup [2, 3]$.} sets. A metric space is \term{connected} if it is not disconnected.
\end{definition}

\begin{definition}{Domain}{}
  A \term{domain} is a connected open set.
\end{definition}

\begin{proposition}{}{}
  Let $X$ be a metric space. Then the following statements are equivalent:
  \begin{romanenum}
    \item $X$ is connected.
    \item If $f: X \to \set{0, 1}$ is continuous, then $f$ is constant.
    \item The only subsets of $X$ which are both open and closed are $X$ and $\emptyset$.
  \end{romanenum}
\end{proposition}
\begin{proof}
  (i) $\implies$ (ii): Let $X$ be connected and $f: X \to \set{0, 1}$ be continuous. Singletons $\set{0}$ and $\set{1}$ are both open in $\set{0, 1}$, so $f^{-1}(0)$ and $f^{-1}(1)$ are both open in $X$. Clearly these two sets are disjoint and their union is $X$, but $X$ is connected so one of them is empty, hence $f$ is constant.
\end{proof}

\begin{proposition}{}{}
  Let $X$ be a metric space and $Y \subseteq X$.\marginnote{Here, $Y$ is considered as a metric space with the metric induced from that of $X$.} Then $Y$ is connected if and only if: for open subsets $U, V \subseteq X$ with $U \cap V \cap Y = \emptyset$, if $Y \subseteq U \cup V$, then either $Y \subseteq U$ or $Y \subseteq V$.
\end{proposition}

\begin{lemma}{Sunflower Lemma}{}
  Let $X$ be a metric space. Let $\set{A_i: i \in I}$ be a collection of connected subsets of $X$ with $\bigcap_{i \in I} A_i \neq \emptyset$. Then $\bigcup_{i \in I} A_i$ is connected.
\end{lemma}
\begin{proof}
  Let $f: \bigcup_{i \in I} A_i \to \set{0, 1}$ be continuous. Pick $x_0 \in \bigcap_{i \in I} A_i$. We know for any $x \in \bigcup_{i \in I} A_i$, there is some $i$ such that $x \in A_i$, then $f$ restricted to $A_i$ is constant as $A_i$ is connected, hence $f(x) = f(x_0)$. Since $x$ was arbitrary, we conclude $f$ is constant. Therefore $\bigcup_{i \in I} A_i$ is connected.
\end{proof}

\begin{lemma}{Connectedness and closures}{}
  Let $X$ be a metric space. If $A \subseteq X$ is connected and $B \subseteq X$ is such that $A \subseteq B \subseteq \closure(A)$, then $B$ is also connected.
\end{lemma}

\begin{lemma}{Connectedness is preserved under continuous maps}{}
  Let $X, Y$ be metric spaces, and $f: X \to Y$ be continuous. Then $f(X)$ is connected.
\end{lemma}

Therefore, unlike completeness, connectedness is preserved under homeomorphisms.

\begin{theorem}{}{}
  A subset of $\RR$ is connected if and only if it is an interval.
\end{theorem}


\subsection{Path Connectedness}
  
  \begin{definition}{Path-connectedness}{}
    Let $X$ be a metric space. We say $X$ is \term{path-connected} if for any $a, b \in X$, there exists a continuous map (path) $\gamma: [0, 1] \to X$ with $\gamma(0) = a,\ \gamma(1) = b$.
  \end{definition}
  
  \begin{theorem}{}{}
    A path-connected metric space is connected.
  \end{theorem}
  
  \begin{theorem}{}{}
    A connected open subset of a normed space is path-connected.
  \end{theorem}
  
  We can see that path-connectedness is a stronger property than connectedness. As a classic example, we can find a connected subset of $\RR^2$ that is not path-connected. The set $A \subseteq \RR^2$, known as topologist's sine-curve, is given by
  \[ A = \set{(0, y): -1 \leqslant y \leqslant 1} \cup \set{\left(x, \sin\frac{1}{x}\right): x \in (0, 1]} \]




\section{Compactness}
  
\subsection{Sequential Compactness}

  We can generalize the Bolzano--Weierstrass property\marginnote{Any bounded sequence of $\RR$ has a convergent subsequence.} of $\RR$ into the following concept for metric spaces.
  
  \begin{definition}{Sequential compactness}{}
    A metric space $X$ is said to be \term{sequentially compact} if any sequence of elements of $X$ has a convergent subsequence.\marginnote{Note that we are defining \emph{sequential compactness} here and not \emph{compactness}. Although they are sometimes used interchangeably, and in fact are equivalent concepts for metric spaces, it is still worth emphasizing the difference here and the nontriviliaty of their equivalence.}
  \end{definition}
  
  \begin{lemma}{}{seq-compact-closed-bounded}
    A sequentially compact subspace of a metric space is closed and bounded.
  \end{lemma}
  
  \begin{lemma}{}{}
    A closed subset of a sequentially compact metric space is also sequentially compact.
  \end{lemma}
  
  \begin{lemma}{}{seq-compact-continuous}
    Sequential compactness is preserved under a continuous map. That is, the image of a sequentially compact metric space under a continuous map is also sequentially compact.
  \end{lemma}
  Combining \cref{lem:seq-compact-continuous} and \cref{lem:seq-compact-closed-bounded}, a corollary is that any continuous map $f: X \to \RR$, where $X$ is sequentially compact, is bounded and has a closed image (i.e.~attains its bounds).
  
  \begin{lemma}{}{}
    The product space of two sequentially compact metric spaces is also sequentially compact.
  \end{lemma}
  
  \begin{theorem}{Bolzano--Weierstrass theorem}{}
    Any closed and bounded subset of $\RR^n$ is sequentially compact.
  \end{theorem}
  
  \begin{lemma}{}{}
    A sequentially compact metric space is complete and bounded.
  \end{lemma}


\subsection{The Arzel\`a--Ascoli Theorem}

  \begin{definition}{Total boundedness}{}
    A metric space $X$ is \term{totally bounded} if for any $\epsilon > 0$, $X$ may be covered by finitely many open balls of radius $\epsilon$.
  \end{definition}
  
  \begin{theorem}{}{}
    A metric space is sequentially compact if and only if it is complete and totally bounded.
  \end{theorem}


\subsection{Compactness}

  To define the notion of compactness, we will first introduce open covers.
  
  \begin{definition}{Covers}{}
    Let $X$ be a metric space and $\mathcal{U} = \set{U_i: i \in I}$ a collection of open subsets of $X$. Then $\mathcal{U}$ is an \term{open cover} of $X$ if $X = \bigcup_{i \in I} U_i$.
    
    If, for some $J \subseteq I$, we have $X = \bigcup_{i \in J} U_i$, then we say $\set{U_i: i \in J}$ is a \term{subcover} of $\mathcal{U}$. Moreover, if $\abs{J}$ is finite, then it is a \term{finite subcover}.
  \end{definition}
  
  \begin{definition}{Compactness}{}
    A metric space is \term{compact} if every open cover of it has a finite subcover.
  \end{definition}
  
  In fact, over metric spaces, compactness and sequential compactness are equivalent.
  
  \begin{theorem}{}{}
    A compact metric space is sequentially compact.
  \end{theorem}
  \begin{proof}
    Let $X$ be a compact metric space and $(x_n)_{n=1}^\infty$ be a sequence in $X$. We want to find a convergence subsequence of this sequence.
    
    For each $n \in \NN$, define $A_n \defeq \set{x_n, x_{n+1}, \dots}$ be a tail of $(x_n)$. We have $A_1 \supseteq A_2 \supseteq \cdots$ and thus $\overline{A_1} \supseteq \overline{A_2} \supseteq \cdots$. Then the intersection $A \defeq \bigcap_{n=1}^\infty \overline{A_n}$ is non-empty.\marginnote{This follows from a lemma which we will not prove here.} Let $a \in A$ and construct a subsequence $(x_{n_k})$ such that for any $k$, we have $d(x_{n_k}, a) < \frac{1}{k}$. This construction can be done inductively as follows: given $x_{n_1}, \dots, x_{n_k}$, since $a \in \overline{A_{n_{k + 1}}}$, this means there is some $x_{n_{k + 1}}$ such that $d(x_{n_{k + 1}}, a) < \frac{1}{k + 1}$. Clearly, this subsequence converges to $a$, hence $X$ is sequentially compact.
  \end{proof}
  
  \begin{theorem}{}{}
    A sequentially compact metric space is compact.
  \end{theorem}
  
  We provide a special case of the above theorem.
  \begin{theorem}{Heine--Borel theorem}{}
    The interval $[a, b]$ is compact.
  \end{theorem}




\newpage
\part{Complex Analysis}

\section{The Complex Plane}

We will skip over the basics of complex numbers and the complex plane.

\subsection{The Extended Complex Plane}
  
  One special property about $\CC$ (compared to, say, $\RR$), is its possibility to be extended to include a point at infinity $\infty$ while preserving nice algebraic properties. This extended complex plane is usually denoted as $\CCx$.
  
  In order to formulate the construction of $\CCx$, we introduce stereographic projection.
  
  \begin{definition}{Stereographic projection}{stereo-proj}
    Let $\mathbb{S}$ denote the unit sphere in $\RR^3$ and $N = (0, 0, 1) \in \mathbb{S}$ denote the `north pole' of $\mathbb{S}$. We define the \term{stereographic projection} which is a map $S: \mathbb{S} \setminus \set{N} \to $ such that for each $s \in \mathbb{S} \setminus \set{N}$, $S(s)$ is the unique point where the line joining $s$ and $N$ intersects the $xy$-plane.\marginnote{We can deduce an explicit formula but it is not necessary here.}
  \end{definition}


\subsection{M\"obius maps}

  M\"obius maps are a special subset of self-maps of $\CCx$. In particular, there is a 1-1 correspondence between the set of M\"obius maps and the general linear group $\GL_2(\CC)$.
  
  \begin{definition}{M\"obius map}{}
    Let $g = \begin{pmatrix} a & b \\ c & d \end{pmatrix} \in \GL_2(\CC)$. Define a M\"obius map $\Psi_g: \CCx \to \CCx$ as \\[-0.5em]
    \[ \Psi_g(z) = \frac{az + b}{cz + d} \]
    where specifically,
    \begin{itemize}
      \item if $c \neq 0$, then define $\Psi_g(-d/c) = \infty$ and $\Psi_g(\infty) = a/c$.
      \item if $c = 0$, then define $\Psi_g(\infty) = \infty$.
    \end{itemize}
  \end{definition}
  
  It is easy to see that M\"obius maps preserves equivalence under non-zero scaling of the matrix $g$. Further, composition of M\"obius maps correspond to multiplication of the corresponding matrices --- therefore, we can say that $\GL_2(\CC)$ acts on $\CCx$ via M\"obius maps.


\subsection{The Complex Projective Line}



\section{Complex Differentiability}

Now we want to transfer concepts and results from real analysis to the complex plane, and introduce arguably one of the most important concepts in complex analysis.

\begin{definition}{Complex differentiability}{complex-diff}
  Let $f: U \to \CC$ be a function on $U$, where $U$ is a neighborhood of $a \in \CC$. Then $f$ is \term{complex differentiable} at $a$ if the limit
  \[ \lim_{z \to a} \frac{f(z) - f(a)}{z - a} \]
  exists, in which case we write $f'(a)$ for the limit. Equivalently,
  \[ f'(a) = \lim_{h \to 0} \frac{f(a + h) - f(a)}{h} \]
  note $h \in \CC$ and we require $a + h \in U$.
\end{definition}

It is easy to verify that all basic properties, including sum, product, quotient rules, and chain rule and power rule, are still satisfied for complex derivatives.

\begin{definition}{Holomorphic function}{holomorphic}
  Let $U \subseteq \CC$ be an open subset of $\CC$, and $f: U \to \CC$ be a function. If $f$ is complex differentiable at every $a \in U$, then we say $f$ is \term{holomorphic} on $U$.\marginnote{Holomorphic is a local property. When one says $f$ is holomorphic at a point $a$, it means there is some neighborhood $U$ of $a$ on which $f$ is holomorphic. A function is \em{never} holomorphic at a single point.}
\end{definition}


\subsection{Cauchy--Riemann Equations}

  Despite the definition looking almost identical to that of real differentiability, complex differentiability is in fact a much more rigid and restrictive property. This can be demonstrated by the following.

  Consider a function $f: U \to \CC$. Since $\RR^2$ and $\CC$ are isomorphic, we can think of $U$ as an open subset of $\RR^2$ and write $f = (u, v)$ such that $u, v: \RR^2 \to \RR$ are \term{components} of $f$ satisfying $f(x + iy) = u(x, y) + iv(x, y)$.
  
  \begin{theorem}{Cauchy--Riemann equations}{c-r}
    Let $U$ be a neighborhood of $a \in \CC$, and let $f: U \to \CC$ be a function which is complex differentiable at $a$. If $u, v$ are the components of $f$, then\marginnote{Note: the converse also holds, but we will not prove it here.} the partial derivatives $\partial_x u, \partial_y u, \partial_x v, \partial_y v$ exist at $a$, and
    \[ \pdv{u}{x} = \pdv{v}{y}, \quad \pdv{v}{x} = -\pdv{u}{y} \]
    Moreover, we have that
    \[ f'(a) = \partial_x u(a) + i\partial_x v(a) \]
  \end{theorem}
  \begin{proof}
    Write $f(z) = f(x + iy) = f(x, y)$ and let $a = a_0 + a_1i$. Taking \cref{def:complex-diff} as using only real $h$, we have
    \[ f'(a) = \frac{f(a_0 + h, a_1) - f(a_0, a_1)}{h} = {\pdv{f}{x}}(a) \]
    Using only imaginary $h$, we have
    \[ f'(a) = \frac{f(a_0, a_1 + h) - f(a_0, a_1)}{ih} = \frac{1}{i} {\pdv{f}{y}}(a) = -i {\pdv{f}{y}}(a) \]
    Therefore we have
    \[ \pdv{f}{x} = -i \pdv{f}{y} \]
    
    Expanding the definition of $f$ using components, we get the existence of the partial derivatives and that
    \begin{align*}
      &\pdv{u}{x} + i\pdv{v}{x} = -i \pdv{u}{y} + \pdv{v}{y} \\
      \implies &\pdv{u}{x} = \pdv{v}{y},\quad \pdv{v}{x} = -\pdv{u}{y} \qedhere
    \end{align*}
  \end{proof}
  
  What this result shows is that complex functions with even very smooth components aren't necessarily complex differentiable. For example, consider $f(z) = \overline{z}$ or $u(x, y) = xy,\ v(x, y) = 0$.
  
  
\subsection{Harmonic Functions}
  \begin{definition}{Harmonic function}{}
    Let $U$ be an open subset of $\RR^2$ and let $u: U \to \RR$ be a twice differentiable function. Define the \term{Laplacian} $\Delta u$ as
    \[ \Delta u = \partial_{xx} u + \partial_{yy} u \]
    Then $u$ is \term{harmonic} if $\Delta u = 0$.
  \end{definition}
  
  The significance of harmonic functions in the context of complex differentiability is demonstrated by the following result.
  
  \begin{theorem}{}{}
    Let $U \subseteq \CC$ be open and let $f: U \to \CC$ be a holomorphic function with components $u, v$. If $u, v$ are both twice continuously differentiable\marginnote{In fact, $f$ being holomorphic already entails it being infinitely differentiable, so this assumption is redundant.}, then $u, v$ are harmonic.
  \end{theorem}
  \begin{proof}
    By the Cauchy--Riemann equations, we have that
    \begin{align*}
      \Delta u
      &= \partial_{xx} u + \partial_{yy} u \\
      &= \partial_x (\partial_y v) + \partial_y (-\partial_x v) \\
      &= \partial_{xy} v - \partial_{yx} v
    \end{align*}
    But $v$ is twice continuously differentiable, so $\Delta u = 0$. By the same reasoning $\Delta v = 0$ as well. So $u, v$ are both harmonic.
  \end{proof}


\subsection{Power Series}
  We will first define a few familiar concepts.
  
  \begin{definition}{}{}
    Let $\sum_{n=0}^\infty a_n z^n$ be a power series, and let $S$ be the (non-empty) set of $z \in C$ at which it converges. The \term{radius of convergence} of this power series is $R = \sup\set{\abs{z}: z \in S}$ (or $\infty$ for unbounded $S$).
  \end{definition}
  
  \begin{proposition}{}{}
    Let $\sum_{n=0}^\infty a_n z^n$ be a power series which converges on $S \subseteq \CC$ and with radius of convergence $R$. Then
    \[ B(0, R) \subseteq S \subseteq \overline{B}(0, R) \]
    Moreover, the series converges absolutely on $B(0, R)$ and converges uniformly on $\overline{B}(0, r)$ for any $0 \leqslant r < R$. Also,
    \[ \frac{1}{R} = \limsup_n \abs{a_n}^{1/n} \]
  \end{proposition}


\subsection{Logarithms}
  
  Logarithms, being the inverse of exponentiation, are problematic when considered in the context of complex numbers. This is because if $e^x = z$, then we have infinitely many $e^{x + 2n\pi i} = z$. We will define an alternative logarithm function.
  
  \begin{proposition}{}{}
    Let $D = \CC \setminus \set{x \in \RR: x \leqslant 0}$. Define the function $\Log: D \to \CC$ such that if $z = \abs{z} e^{i\theta}$ for $\theta \in (-\pi, \pi]$, then $\Log(z) \defeq \log\abs{z} + i\theta$. We have that, for any $z \in D$,
    \begin{romanenum}
      \item $\exp(\Log(z)) = z$;
      \item $\Log$ is holomorphic on $D$ (and hence at $z$);
      \item $\Log'(z) = \frac{1}{z}$. \marginnote{The proof of this is rather technical and is therefore disregarded}
    \end{romanenum}
  \end{proposition}



\section{Branch Cuts}
  
  Usually, a holomorphic function on a domain $D \subseteq \CC$ does not extend into the whole complex plane. For example, consider the multi-valued function $f(z) = z^{1/2}$. If we take $f(re^{i\theta}) = r^{1/2} e^{i\theta/2}$ where $\theta \in [0, 2\pi)$, then it is only holomorphic on the complex plane without the positive real axis (i.e. the set $\CC \setminus \set{z \in \CC: \Im(z) = 0, \Re(z) > 0}$). In this case, the positive real axis is a branch cut for the function $f$. By picking the negative function $f(re^{i\theta}) = -r^{1/2} e^{i\theta/2}$, or by choosing different intervals for the argument (e.g. $(-\pi, \pi]$ instead), we can take different cuts and obtain different branches.
  
  \begin{definition}{}{}
    A \term{multi-valued function} (or \term{multifunction}) on $U \subseteq \CC$ is a map $f: U \to \powerset(\CC)$ from each point in $U$ to a subset of $\CC$. A \term{branch} of $f$ on some $V \subseteq U$ is a function $g: V \to \CC$ with $\forall z \in V: g(z) \in f(z)$.
  \end{definition}
  
  \begin{definition}{}{}
    Let $f: U \to \powerset(\CC)$ be a multifunction on $U \subseteq \CC$. A point $z_0 \in U$ is a \term{branch point} of $f$ if there doesn't exist an open disk $D \subseteq U$ containing $z_0$ such that there is a holomorphic branch of $f$ defined on $D \setminus \set{z_0}$.
    
    A \term{branch cut} for $f$ is a curve in the plane on whose complement we can pick a holomorphic branch of $f$. Therefore, a branch cut must contain all branch points.
  \end{definition}
  
  Remark that when $\CC \setminus U$ is bounded, we say $f$ has a branch point at $\infty$ if there is no holomorphic branch of $f$ defined on $\CC \setminus B(0, R) \subseteq U$ for any $R > 0$.



\section{Paths and Integration}

\subsection{Paths}
  
  \begin{definition}{Path}{}
    A \term{path} in the complex plane is a continuous function $\gamma: [a, b] \to \CC$, and we will denote $\gamma^*$ for its image. We also define its \term{opposite path} $\gamma^-: [a, b] \to \CC$ as $\gamma^-(t) = \gamma(a + b - t)$.
    
    For a path $\gamma$, we say $\gamma$ is \term{closed} if $\gamma(a) = \gamma(b)$; we say $\gamma$ is \term{simple} if $\gamma$ is not self-intersecting --- that is, we have $\gamma(s) \neq \gamma(t)$ for any $s \neq t$.
  \end{definition}
  
  \begin{definition}{Path concatenation}{}  
    Two paths $\gamma_1: [a, b] \to \CC$ and $\gamma_2: [c, d] \to \CC$ satisfying $\gamma_1(b) = \gamma_2(c)$ can be \term{concatenated} into $\gamma_1 \concat \gamma_2: [a, b - c + d] \to \CC$ where
    \[ \gamma_1 \concat \gamma_2(t) = \begin{cases}
      \gamma_1(t), & t \leqslant b \\
      \gamma_2(t - b + c), & t > b
    \end{cases} \]
  \end{definition}
  
  \begin{definition}{Smooth path}{}
    A path $\gamma$ is \term{$C^1$} (or \term{smooth}) if it is differentiable (i.e. both its real and imaginary parts are differentiable) and its derivative $\gamma'(t)$ is continuous.\marginnote{While a jump discontinuity is not permitted in the derivative, an essential discontinuity could exist.}
    
    A path $\gamma: [a, b] \to \CC$ is \term{piecewise-$C^1$} (or \term{piecewise-smooth}) if there exists a partition $a = a_0 < a_1 < \dots < a_n = b$ such that $\gamma$ is $C^1$ on each interval $[a_{i-1}, a_i]$ for $i = 1, \dots, n$.
    
    A \term{contour} is a simple, closed, piecewise-$C^1$ path.\marginnote{It is worth noting that some authors don't require contours to be simple.}
  \end{definition}
  
  \begin{definition}{Path reparametrization}{path-reparametrization}
    Let $\phi: [a, b] \to [c, d]$ be continuously differentiable with $\phi(a) = c, \phi(b) = d$, and let $\gamma: [c, d] \to \CC$ be a $C^1$ path. A \term{reparametrization} $\tilde{\gamma}: [a, b] \to \CC$ of $\gamma$ is defined by $\tilde{\gamma} = \gamma \comp \phi$. $\tilde{\gamma}$ is also $C^1$ and has the same image as $\gamma$.
  \end{definition}
  
  We now define homotopy, or equivalence, of paths. Intuitively, two paths are equivalent if we can continuously merge one into another without leaving the space in consideration, keeping the endpoints fixed.
  
  \begin{definition}{Path equivalence}{path-homotopy}
    Two paths $\gamma_1: [a, b] \to \CC$ and $\gamma_2: [c, d] \to \CC$ are \term{equivalent} if there exists a continuously differentiable bijective function $s: [a, b] \to [c, d]$ with $\gamma_1 = \gamma_2 \comp s$ and $\forall t \in [a, b]: s'(t) > 0$ (to ensure paths are traversed in the same direction). We denote the equivalence class of $\gamma$ by $[\gamma]$, and call the equivalence classes \term{oriented curves}.
  \end{definition}


\subsection{Integration Along A Path}
  
  We first look at integrating a complex-valued function $F$ along an interval $[a, b]$. We can write $F(t) = G(t) + iH(t)$ for real functions $G, H$. Then $F$ is Riemann-integrable iff both $G, H$ are, and
  \[ \int_a^b F(t) \dd{t} \defeq \int_a^b G(t) \dd{t} + i\int_a^b H(t) \dd{t} \]
  
%  In addition to the linearity of this integral, we also have the following:
%  \begin{lemma}{}{}
%    Let $F: [a, b] \to \CC$ be a complex function. Then
%    \[ \abs{\int_a^b F(t) \dd{t}} \leqslant \int_a^b \abs{F(t)} \dd{t} \]
%  \end{lemma}
  
  \begin{definition}{Integration along a path}{}
    Let $F: \CC \to \CC$ and $\gamma: [a, b] \to \CC$ be a $C^1$ path. We define the integral of $f$ along $\gamma$ to be
    \[ \int_\gamma f(z) \dd{z} = \int_a^b f\left(\gamma(t)\right) \gamma'(t) \dd{t} \]
    For a piecewise-$C^1$ path, we define the integral to be the sum of integrals along each smooth pieces:
    \[ \int_\gamma f(z) \dd{z} = \sum_{i=1}^n \int_{a_{i-1}}^{a_i} f\left(\gamma(t)\right) \gamma'(t) \dd{t} \]
  \end{definition}
  
  It suffices that $f(\gamma(t))$ and $\gamma'(t)$ to be bounded and continuous almost everywhere\marginnote{This means we allow finitely many exceptions} for the integral to exist, but note that $\gamma'(t)$ actually already fulfills this requirement from its $C^1$ assumption. Therefore, we only have to check $f(\gamma(t))$.
  
  Importantly, we note that the value of the integral only depends on the oriented curves and not the specific parametrization, we thus have the following result.
  \begin{lemma}{}{}
    Let $\gamma: [a, b] \to \CC$ and $\tilde{\gamma}: [c, d] \to \CC$ be equivalent piecewise-$C^1$ paths. Let $f: \CC \to \CC$ be continuous, then
    \[ \int_\gamma f(z) \dd{z} = \int_{\tilde{\gamma}} f(z) \dd{z} \]
  \end{lemma}
  
  In addition, the length of a path is also dependent only on oriented curves and remains constant for equivalent paths (invariant to $C^1$-reparametrization). Additionally, lengths are also invariant to the direction of the curves, so $\ell(\gamma) = \ell(\gamma^-)$.
  
  \begin{definition}{}{}
    Let $\gamma: [a, b] \to \CC$ be a $C^1$ path. The \term{length} of $\gamma$ is
    \[ \ell(\gamma) \defeq \int_a^b \abs{\gamma'(t)} \dd{t} \]
  \end{definition}
  
  Apart from the linearity of integration along a path, we also have the following properties,
  \begin{proposition}{}{}
    Let $f: U \to \CC$ be a continuous function on an open $U \subseteq \CC$, and let $\gamma, \eta: [a, b] \to \CC$ be piecewise-$C^1$ paths whose image is in $U$. Then,
    \begin{romanenum}
      \item $\displaystyle \int_\gamma f(z) \dd{z} = -\int_{\gamma^-} f(z) \dd{z}$
      \item $\displaystyle \int_{\gamma \concat \eta} f(z) \dd{z} = \int_\gamma f(z) \dd{z} + \int_\eta f(z) \dd{z}$ \hfill (additivity)
      \item $\displaystyle \abs{\int_\gamma f(z) \dd{z}} \leqslant \sup_{z \in \gamma^*} \abs{f(z)} \cdot \ell(\gamma)$ \hfill (Estimation Lemma)
    \end{romanenum}
  \end{proposition}
  
\subsection{Primitives}
  
  \begin{definition}{}{}
    Let $U$ be an open subset of $\CC$ and let $f: U \to \CC$ be a continuous function. A \term{primitive} for $f$ on $U$ is a differentiable function $F: U \to \CC$ such that $\forall z \in U: F'(z) = f(z)$.
  \end{definition}
  
  The fundamental theorem of calculus applies to complex integrals.
  \begin{theorem}{Fundamental theorem of calculus}{fundamental-thm-calc}
    Let $U$ be an open subset of $\CC$ and $F: U \to \CC$ be a primitive for a continuous function $f: U \to \CC$, and let $\gamma: [a, b] \to \CC$ be a piecewise-$C^1$ path in $U$. Then,
    \[ \int_\gamma f(z) \dd{z} = F(\gamma(b)) - F(\gamma(a)) \]
  \end{theorem}
  \begin{proof}
    Suppose $\gamma$ is $C^1$. Then
    \begin{align*}
      \int_\gamma f(z) \dd{z} &= \int_\gamma F'(z) \dd{z} \\
      &= \int_a^b F'(\gamma(t)) \gamma'(t) \dd{t} \\
      &= \int_a^b \dv{t} F(\gamma(t)) \dd{t} \\
      &= F(\gamma(b)) - F(\gamma(a))
    \end{align*}
    
    Now suppose $\gamma$ is piecewise-$C^1$ with partition $a = a_0 < \dots < a_n = b$, then we have a telescoping sum
    \begin{align*}
      \int_\gamma f(z) \dd{z} &= \sum_{i=1}^n \int_{a_{i-1}}^{a_i} f(\gamma(t)) \gamma'(t) \dd{t} \\
      &= \sum_{i=1}^n F(\gamma(a_i)) - F(\gamma(a_{i-1})) \\
      &= F(\gamma(b)) - F(\gamma(a)) \qedhere
    \end{align*}
  \end{proof}
  Hence, importantly, note that the integral of such a function along any closed path will be $0$. Importantly, this also suggests that not all holomorphic functions have a primitive.\marginnote{As a canonical example, the holomorphic function $f(z) = \frac{1}{z}$ integrates over the unit circle to $2\pi i$ instead of $0$.}
  
  We also have the following result regarding the existence of a primitive, which is roughly the converse of the above \cref{thm:fundamental-thm-calc}.
  \begin{theorem}{}{}
    Let $U$ be a domain\marginnote{i.e. open and path-connected subset} and $f: U \to \CC$ be a continuous function such that for any closed path $\gamma$ in $U$ we have $\int_\gamma f(z) \dd{z} = 0$. Then $f$ has a primitive.
  \end{theorem}



\section{Winding Numbers}

\begin{definition}{Winding number}{}
  Let $\gamma: [0, 1] \to \CC \setminus \set{0}$ be a closed path, and let $a: [0, 1] \to \RR$ be such that $\gamma(t) = \abs{\gamma(t)} e^{2\pi i a(t)}$. The \term{winding number} $I(\gamma, 0) \in \ZZ$ of $\gamma$ around $0$ is\marginnote{The winding number is guaranteed to be an integer because $\gamma(0) = \gamma(1)$.}
  \[ I(\gamma, 0) = a(1) - a(0) \]
  For any $z_0 \notin \gamma^*$, we define $t: \CC \to \CC$ by $t: z \mapsto z - z_0$ and define
  \[ I(\gamma, z_0) = I(t \comp \gamma, 0) \]
\end{definition}

\begin{definition}{Analytic function}{}
  Let $f: U \to \CC$ be a function on an open subset $U$ of $\CC$. We say $f$ is \term{analytic} on $U$ if for every $z_0 \in U$ there is $r > 0$ with $B(z_0, r) \subseteq U$ such that there is a power series $\sum_{k=0}^\infty a_k (z-z_0)^k$ with radius of convergence at least $r$ and $\sum_{k=0}^\infty a_k (z-z_0)^k = f(z)$.
\end{definition}

Note that any analytic function is holomorphic since any power series is infinitely differentiable.



\section{Cauchy's Theorem}

\subsection{Goursat's Theorem}

  We start by proving the version of Cauchy's theorem with the simplest closed contours --- triangles, which is the concatenation of 3 linear paths. We can also generalize this result to any polygonal paths.

  \begin{theorem}{Goursat's Theorem}{}
    Let $U$ be an open subset of $\CC$, and $T \subseteq U$ be a triangle whose interior is also contained in $U$. Then, for any function $f$ holomorphic in $U$, we have
    \[ \int_T f(z) \dd{z} = 0 \]
  \end{theorem}
  
  This leads to the following result, which suggests that \em{locally}, every holomorphic function has a primitive.
  \begin{theorem}{}{}
    A holomorphic function in an open disc has a primitive in that disc.
  \end{theorem}
  
  \begin{theorem}{Cauchy's theorem for discs}{}
    If $f: U \to \CC$ is holomorphic in a disc, then for any closed curve $\gamma$ in that disc, we have
    \[ \int_\gamma f(z) \dd{z} = 0 \]
  \end{theorem}

\begin{theorem}{Morera's Theorem}{}
  Suppose $f: U \to \CC$ is a continuous function on an open subset $U \subseteq \CC$. If for any closed path $\gamma: [a, b] \to U$ there is $\int_\gamma f(z) \dd{z} = 0$, then $f$ is holomorphic.
\end{theorem}


\subsection{Cauchy's Integral Formula}

  \begin{theorem}{Cauchy's integral formula}{}
    Let $U \subseteq \CC$ be an open set containing a closed disc $\overline{B}(a, r)$. Let $f: U \to \CC$ be a holomorphic function, then for any $w \in B(a, r)$,
    \[ f(w) = \frac{1}{2\pi i} \int_\gamma \frac{f(z)}{z - w} \dd{z} \]
    where $\gamma: [0, 1] \to \CC$ is the closed path $t \mapsto a + re^{2\pi it}$
  \end{theorem}
  \begin{proof}
    
  \end{proof}
  
  \subsubsection{Applications of Cauchy's Integral Formula}
  
    \begin{theorem}{}{}
      Let $f: U \to \CC$ be a holomorphic function on an open set $U \subseteq \CC$. Then for any $z_0 \in U$, we know $f(z)$ is equal to its Taylor series evaluated at $z_0$
      The derivatives of $f$ are given by
      \[ f^{(n)}(z_0) = \frac{n!}{2\pi i} \int_{\gamma(a, r)} \frac{f(z)}{(z - z_0)^{n+1}} \dd{z} \]
    \end{theorem}
    
  
  \begin{example}{}{cauchy-integral-formula}
    \[ \int_0^\infty \frac{1 - \cos x}{x^2} \dd{x} = \frac{\pi}{2} \]
  \end{example}
  \begin{proof}
    Consider the function $f: \CC \to \CC$ defined by $f(z) = \frac{1 - \exp(iz)}{z^2}$. Define the path $\gamma_R$ to be the upper semicircle centered at 0 with radius $R \gg 0$ and positive orientation, and the path $\gamma_\epsilon$ to be the upper semicircle centered at 0 with radius $\epsilon > 0$ and negative orientation.
    
    \begin{align*}
      &\int_\gamma f(z) \dd{z} \\
      = &\int_{-R}^{-\epsilon} \frac{1 - \exp(ix)}{x^2} \dd{x} + \int_{\gamma_\epsilon} f(z) \dd{z} + \int_{\epsilon}^{R} \frac{1 - \exp(ix)}{x^2} \dd{x} + \int_{\gamma_R} f(z) \dd{z} \\
      = &\ 0
    \end{align*}
    
    As $R \to \infty$, we have
    \[ \abs{\frac{1 - \exp(iz)}{z^2}} \leqslant \frac{2}{\abs{z}^2} \]
  \end{proof}





\section{Identity Theorem}

\begin{theorem}{Identity Theorem}{}
  Let $U$ be a domain and $f_1, f_2$ be holomorphic functions defined on $U$. If the set $S = \set{z \in U: f_1(z) = f_2(z)}$ has a limit point in $U$, then $S = U$, that is, $\forall z \in U: f_1(z) = f_2(z)$.
\end{theorem}
\begin{proof} % TODO
  Let $g = f_1 - f_2$ which is also holomorphic in $U$, then $S = g^{-1}(\set{0})$. Since $g$ is holomorphic, XXX, any $z_0 \in S$ is either an isolated point or lies in an open ball contained in $S$. Therefore, we can write $S = T \cup V$ where $T$ is the set of isolated points in $S$ and $V = \interior S$ is open.
  
  $g$ is also continuous, so $S = T \cup V$ is closed in $U$ (since $\set{0}$ is closed). Therefore, $\closure_U(V)$, the closure of $V$ in $U$, lies in $V \cup T$.
  
  However, by definition, no limit point of $V$ can lie in $T$, hence $\closure_U(V) = V$, thus $V$ is both open and closed in $U$. Further, by the connectedness of $U$, we conclude that $V = \emptyset$ or $V = U$. In the former case, $T = U$ and all zeros of $g$ are isolated, so $S' = T' = \emptyset$ and $S$ has no limit points. In the latter case, $V = S = U$ as desired.
\end{proof}


\subsection{Isolated Singularities}
  
  \begin{definition}{Isolated singularity}{}
    Let $f: U \to \CC$ be a function and $z_0 \in U$. If $f$ is holomorphic on $B(z_0, r) \setminus \set{z_0}$ for some $r > 0$ but not defined or not holomorphic at $z_0$, then we say $z_0$ is an \term{isolated singularity} of $f$.
  \end{definition}
  
  There are 3 types of isolated singularity: removable singularities, poles, and essential singularities. There are couple of methods to determine the type of an isolated singularity, we will first provide their basic concepts.
  \begin{itemize}
    \item If $f$ is bounded near $z_0$, then we can redefine $f(z_0)$ to make $f$ holomorphic at $z_0$; this makes $z_0$ a \term{removable singularity}.
    \item If $f$ is not bounded near $z_0$, but the function $1 / f(z)$ has a removable singularity at $z_0$, then $z_0$ is a \term{pole}. In this case, we can write $(1/f)(z) = (z-z_0)^m g(z)$ where $g(z_0) \neq 0$, and we call $m$ the \term{order} of the pole at $z_0$. If $m = 1$, the pole is \term{simple}.
    \item Otherwise, $z_0$ is an \term{essential singularity}.
  \end{itemize}
  
  \begin{definition}{Meromorphic function}{}
    Let $U \subseteq \CC$ be open and $f: U \to \CC$ be a function. If all the isolated singularities of $f$ are poles, then we say $f$ is \term{meromorphic} on $U$.
  \end{definition}
  
  
\subsection{Laurent Series}
  
  A corollary of the above definition for poles is that we can express a holomorphic function $f$ around a pole by a generalized version of Taylor series, where the power starts from negative integers instead of $0$. This fact can be generalized to all isolated singularities, as shown below.
  
  \begin{theorem}{Laurent's Theorem}{laurent}
    Let $U \subseteq \CC$ be open and $f: U \to \CC$ be holomorphic with an isolated singularity at $z_0$, then there is a neighborhood $B(z_0, r)$ such that for any $z \in B(z_0, r) \setminus \set{z_0}$, we have the following \term{Laurent series}
    \[ f(z) = \sum_{n=-\infty}^\infty c_n (z - z_0)^n \]
  \end{theorem}
  
  \begin{definition}{}{}
    Continuing the above, if we separate the Laurent series into two parts,
    \[ f(z) = \sum_{n=-\infty}^{-1} c_n (z - z_0)^n + \sum_{n=0}^\infty c_n (z - z_0)^n \]
    the first part with all negative powers is called the \term{principal part} of $f$ at $z_0$, which we will denote by $P_{z_0}(f)$, and the second part with all non-negative powers is called the \term{analytic part} of $f$ at $z_0$. The coefficient $c_{-1}$ is the \term{residue} of $f$ at $z_0$, which we will denote by $\residue_{z_0}(f)$.\marginnote{The name `residue' and its importance will soon become clearer when we discuss the residue theorem}
  \end{definition}
  
  We can also use the Laurent series to determine the type of the isolated singularity by the number of non-zero terms in the principal part. We summarize the method and another method regarding limits in the table below.
    
  \begin{table}[h]
  \centering
  \begin{tabular}{l|l|l}
    \textbf{Type} & \textbf{Principal part} $\bm{P_{z_0}(f)}$ & $\bm{\lim_{z \to z_0} f(z)}$ \\ \hline
    Removable & none & finite \\
    Pole & finitely many terms & $\infty$ \\
    Essential & infinitely many terms & DNE
  \end{tabular}
  \end{table}
  
  \begin{theorem}{Casorati--Weierstrass}{}
    Let $U \subseteq \CC$ be open and $z_0 \in U$. Let $f: U \setminus \set{z_0} \to \CC$ be a holomorphic function with an isolated singularity at $z_0$. Then for all $r > 0$ with $B(z_0, r) \subseteq U$, we have that the set $f\left(B(z_0, r) \setminus \set{z_0}\right)$ is dense in $\CC$, that is, its closure is all of $\CC$.
  \end{theorem}
  \begin{proof}
    AFSOC there is some $r > 0$ with $B(z_0, r) \subseteq U$ such that some $z \in \CC$ is not a limit point of $f\left(B(z_0, r) \setminus \set{z_0}\right)$. This means the function $g(z) = 1/(f(z) - z_0)$ is bounded and non-vanishing on $B(z_0, r) \setminus \set{z_0}$ XXXX
  \end{proof}
  
  \begin{lemma}{}{}
    Suppose $f: U \to \CC$ has a pole of order $m$ at $z_0$, then\marginnote{This is easily derivable from the definition of poles. It is most useful for simple poles since no derivatives are required.}
    \[ \residue_{z_0}(f) = \lim_{z \to z_0} \frac{1}{(m - 1)!} \left(\dv{z}\right)^{\hspace{-0.25em}m-1} (z - z_0)^m f(z) \]
  \end{lemma}




\section{Homotopies}

In this section, we focus on two methods of formalizing the interior of a curve --- the first using homotopies and the second using winding numbers.

\subsection{Homotopies}

  \begin{definition}{Homotopy}{}
    Let $U \subseteq \CC$ be open and $a, b \in U$. Suppose $\eta, \gamma: [0, 1] \to U$ are paths in $U$ with $\eta(0) = \gamma(0) = a$ and $\eta(1) = \gamma(1) = b$. Then $\eta$ and $\gamma$ are \term{homotopic} if there is a continuous function $h: [0, 1] \times [0, 1] \to U$ such that\marginnote{Intuitively, the first argument gives a path somewhere between $\eta$ and $\gamma$, and the second argument controls the actual progression in the path. The function $h$ sort of `sweeps across' the two paths, or deforms one into the other.}
    \[ \begin{cases}
      h(s, 0) = \gamma(t), \quad h(s, 1) = \eta(t) \\
      h(0, t) = a, \quad h(1, t) = b
    \end{cases} \]
  \end{definition}
  
  \begin{definition}{Simply-connected domain}{}
    A domain $U$ in $\CC$ is \term{simply-connected} if for every $a, b \in U$, any two paths from $a$ to $b$ are homotopic in $U$.\marginnote{We can also think of this as: in a simply-connected domain, any closed curve can be continuously deformed into a single point while staying in the domain.}
  \end{definition}
  
  Apart from the obvious examples such as disks, ellipsoids, we also have that the interior of keyhole contours are simply-connected. This is due to the following facts: (1) convex sets are simply-connected; (2) a keyhole contour's interior is the image of the convex set $(\eta, 1) \times (0, 1 - \epsilon)$ under the (homeomorphic) map $(r, \theta) \mapsto re^{2\pi i\theta}$; (3) simply-connectedness is preserved under homeomorphisms.
  
  We can now give an extension to Cauchy's theorem.
  
  \begin{theorem}{Cauchy's theorem, extended}{}
    Let $U$ be a domain in $\CC$ and $a, b \in U$. Let $f: U \to \CC$ be holomorphic, then if paths $\eta, \gamma$ from $a$ to $b$ are homotopic, we have\marginnote{In particular, if $U$ is simply-connected, then the result automatically holds.}
    \[ \int_\eta f(z) \dd{z} = \int_\gamma f(z) \dd{z} \]
  \end{theorem}
  
  Note that if a closed path $\gamma$ is homotopic to a constant path (which is the case in simply-connected domains), then this is reduced to the original version of Cauchy's theorem which says the integral vanishes.


\subsection{Winding Numbers}
  
  \begin{definition}{Annulus}{}
    Let $r, R \in \RR$ with $R > r \geqslant 0$, and let $z_0 \in \CC$. Define an \term{open annulus} as
    \[ A(r, R, z_0) = B(z_0, R) \setminus \overline{B}(z_0, r) = \set{z \in \CC: r < \abs{z - z_0} < R} \]
  \end{definition}



\section{The Residue Theorem}

Cauchy's theorem tells us the integral over a closed path in an open set is $0$ when the function is holomorphic inside that set. We now consider what happens when the function has poles in the interior of the path.

\begin{theorem}{Residue theorem}{}
  Let $U$ be an open subset of $\CC$ and $\gamma$ be a path whose interior is contained in $U$. Let $f$ be a holomorphic function on $U \setminus S$ where $S$ is a finite set such that $S \cap \gamma^* = \emptyset$ (i.e. no points in $S$ lie on the image of $\gamma$). Then,
  \[ \int_\gamma f(z) \dd{z} = 2\pi i \sum_{a \in S} I(\gamma, a) \residue_a(f) \]
\end{theorem}
%\begin{proof}
%  
%\end{proof}

\begin{example}{}{}
  \[ \int_0^{2\pi} \frac{\dd{t}}{1 + 3\cos^2 t} = \pi \]
\end{example}
\begin{proof}
  Similar to \cref{ex:cauchy-integral-formula}, we let $z \defeq e^{it}$, so $\dd{z} = iz \dd{t}$, and
  \begin{align*}
    &\cos t = \real(z) = \frac{1}{2} (z + \overline{z}) = \frac{1}{2} \left(z + \frac{1}{z}\right) \\
    \implies &\frac{1}{1 + 3\cos^2 t} = \frac{1}{1 + \frac{3}{4}\left(z^2 + 2 + \frac{1}{z^2}\right)} = \frac{4z^2}{3 z^4 + 10 z^2 + 3}
  \end{align*}
  Also, define a path $\gamma: [0, 2\pi] \to \CC,\ t \mapsto e^{it}$, then
  \[ \int_0^{2\pi} \frac{\dd{t}}{1 + 3\cos^2 t} = \int_\gamma \frac{4z^2}{3 z^4 + 10 z^2 + 3} \frac{1}{iz} \dd{z} = \int_\gamma \frac{-4iz}{3 z^4 + 10 z^2 + 3} \dd{z} \]
  
  Now that we have the contour integral, denote the final integrand by the function $g$. We want to find the residues of $g$ and then apply the residue theorem to obtain the result. The denominator has zeros $z^2 \in \set{-3, -1/3}$, since $\gamma$ is the unit circle, we only consider the poles $\pm i/\sqrt{3}$. Note since the denominator has degree $4$ and has $4$ distinct zeros, the poles must all be simple, therefore, the residues can be obtained by
  \begin{flalign*}
    &&\residue_{\pm i/\sqrt{3}}(g) &= \lim_{z \to \pm i/\sqrt{3}} \left(z \mp i/\sqrt{3}\right) g(z) \\
    && &= \lim_{z \to \pm i/\sqrt{3}} \frac{-4iz (z \mp i/\sqrt{3})}{3 z^4 + 10 z^2 + 3} \\
    && &= \lim_{z \to \pm i/\sqrt{3}} \frac{-4i(z \mp i/\sqrt{3}) - 4iz}{12z^3 + 20z} && \text{[L'H\^opital]} \\
%    && &= \lim_{z \to \pm i/\sqrt{3}} \frac{-2iz \pm 1/\sqrt{3}}{3z^3 + 5z} \\
    && &= \frac{(\pm 1/\sqrt{3} \mp 1/\sqrt{3}) \pm 1/\sqrt{3}}{3(\pm i/\sqrt{3})^3 + 5(\pm i/\sqrt{3})} \\
    && &= \frac{\pm 1}{\mp i \pm 5i} \\
    && &= \frac{1}{4i}
  \end{flalign*}
  Finally, we can apply the residue theorem to get
  \[ \int_0^{2\pi} \frac{\dd{t}}{1 + 3\cos^2 t} = 2\pi i \left( \residue_{i/\sqrt{3}}(g) + \residue_{-i/\sqrt{3}}(g) \right) = \pi \qedhere \]
\end{proof}


\subsection{Jordon's Lemma}
  
  \begin{lemma}{Jordon's Lemma}{}
    Let $f: \mathbb{H} \to \CCx$ be a meromorphic function on the upper-half plane $\mathbb{H} = \set{z \in \CC: \Im z > 0}$. Suppose $f(z) \to 0$ as $z \to \infty$ in $\mathbb{H}$. Then let $\gamma_R: [0, \pi] \to \CC,\ t \mapsto Re^{it}$ and for any $\alpha > 0$ we will have
    \[ \lim_{R \to \infty} \int_{\gamma_R} f(z) e^{i\alpha z} \dd{z} = 0 \]
  \end{lemma}
  
  \begin{lemma}{Indentation Lemma}{}
    Let $f: U \to \CC$ be a meromorphic function with exactly one simple pole at $a \in U$, and for some $\epsilon > 0$, let $\gamma_\epsilon: [\alpha, \beta] \to \CC,\ t \mapsto a + \epsilon e^{it}$. Then,
    \[ \lim_{\epsilon \to 0} \int_{\epsilon_\epsilon} f(z) \dd{z} = i (\beta - \alpha) \residue_a(f) \]
  \end{lemma}


\subsection{Computing Residues}
  
  We will discuss a method used to compute the residue of a function $F: U \to \CC$ which is the ratio of two holomorphic functions.
  
  \begin{example}{}{}
    
  \end{example}



\section{The Argument Principle}

\begin{theorem}{Argument principle}{}
  Let $U \subseteq \CC$ be open and $f: U \to \CC$ be a meromorphic function. If $B(a, r) \subseteq U$ for some $a \in \CC, r \in \RR$, and let $N, P$ be the number of zeros and poles (both counted with multiplicity) of $f$ inside $B(a, r)$, and suppose $f$ has neither on $\partial B(a, r)$, then,
  \[ N - P = \frac{1}{2\pi i} \int_\gamma \frac{f'(z)}{f(z)} \dd{z} \]
  where $\gamma: [0, 1] \to \CC,\ t \mapsto a + re^{2\pi it}$ is a path describing $\partial B(a, r)$. Moreover, $N - P$ is also the winding number of the path $\Gamma = f \circ \gamma$ about the origin.
\end{theorem}



\section{Conformal Transformations}

Recall the stereographic projection map we introduced back in \cref{def:stereo-proj}. One important property of it is that it is conformal.

Roughly, a conformal map is one that preserves angles locally.

\begin{definition}{Conformal map}{}
  
\end{definition}

\begin{proposition}{}{}
  Let $f: U \to \CC$ be a holomorphic map with $f'(z_0) \neq 0$ for some $z_0 \in U$. Then $f$ is conformal at $z_0$.
  
  More generally, if $f$ has non-vanishing derivative on all of $U$, then it is conformal on all of $U$.
\end{proposition}
\begin{proof}
  
\end{proof}

\begin{proposition}{}{}
  The stereographic projection map $S: \CC \to \mathbb{S}$ is conformal.
\end{proposition}

\begin{proposition}{}{}
  All M\"obius tranformations are conformal (except at the pole).
\end{proposition}

An important result is that M\"obius transformations can be uniquely identified by a triple (and their image).

\begin{theorem}{Riemann's mapping theorem}{}
  Let $U$ be an open, connected, and simply-connected proper\marginnote{By Liouville's theorem, we can immediately see that there exists no bijective conformal transformation from $\CC$ to $B(0, 1)$.} subset of $\CC$. Then for any $z_0 \in U$, there is a unique bijective conformal transformation $f: U \to B(0, 1)$ such that $f(z_0) = 0$ and $f'(z_0) > 0$.
\end{theorem}





\end{document}
